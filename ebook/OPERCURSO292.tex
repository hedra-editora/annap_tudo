\part{O percurso} 

\epigraph{Quando mergulhamos em nós mesmos, não descobrimos uma personalidade
autônoma desvinculada de momentos sociais, mas as marcas do sofrimento
do mundo alienado.}{Theodor Adorno} 

\chapter{Onipotência}

Quando vi aquela mulher nua no hospital, desejei fugir correndo. O~corpo
dela parecia o meu, mas arruinado pela dor e pelo tempo. Eu já tinha
tirado sua camisola suja, a calcinha bege e encarado o repulsivo
absorvente urinário. Para lhe dar banho, tentei me convencer de que o
bebê não se torna abjeto por não conter as próprias fezes. Ela chorava
assustada. Fiz o trabalho com nojo.

Normalmente, a doentinha era eu --- não minha mãe. Quando eu ainda era
uma menininha, ela permitia que o irmão mais velho, mais gordo e mais
forte me batesse diariamente: dizia que eu deveria aprender a me
defender. Também dizia: ``Só você pode educar seu pai''. Por isso a
menininha gostava muito de ficar doente. Era quando a mamãe aparecia e a
deixava fraquejar. Daí vinham os analgésicos, antitérmicos,
anti-inflamatórios, antibióticos, antialérgicos, antiespasmódicos,
antiácidos, anti-histamínicos, antidepressivos, ansiolíticos…

Naquela semana de hospital, tentei administrar todos aqueles antídotos
ao corpo abatido de minha mãe. Ela chorava, gemia e tinha medo. Achava
que não podia adoecer. Tinha acreditado que poderia nunca adoecer. Para
curá-la, segurei sua mão, controlei os remédios, dei comida, banho e
protegi seu sono. No fim de tudo, ambas estávamos curadas.

\chapter{Vulnerabilidade}

\mbox{}\indent{}--- Ah, Anna! Deixa o moço tirar uma casquinha… --- foi o que
disse minha mãe quando me queixei do assédio do oftalmologista. Minha
mãe era uma profissional bem sucedida, uma intelectual brilhante. ---
Deixa ele tirar uma casquinha. Quem mandou ser tão bonitinha? O que é
que custa? Já tá aí mesmo. Tá se achando a última bolacha do pacote? A
rainha da cocada preta? Isso tudo um dia acaba e você nem aproveitou.
Orgulhosa. Coitado do moço! Você não é melhor do que ninguém!

--- Você é que era muito erotizada! --- berrou meu pai no restaurante.
Ele gostava muito de berrar. Ao contrário do que eu dizia, ele nunca
havia me assediado. Se toda vez que me encontrava ele comentava casos de
pais que estupraram as filhas, era só para mostrar como ele era um bom
pai. Afinal, ele poderia ter me estuprado, mas não estuprou. Afinal,
quando eu era bem pequenininha, eu me masturbava na frente deles. Eu era
muito erotizada. E~depois virei uma mulher muito bonita. E~o que é que
tinha se sempre que saíamos de carro ele ficava apertando as minhas
coxas? Carinho de pai. E~minhas coxas eram muito bonitas. Muito
erotizadas. E~o que é que tinha se ele gostava de sair sempre sozinho
comigo para os outros pensarem que eu era sua namorada? Eu era tão
bonita! Tão erotizada! E o que tinha de mais se a namorada psicanalista
dele sempre dizia que a nossa relação era muito erotizada? Eu é que era
muito erotizada!

\chapter{Desamparo}

Anna,

\smallskip{} 

Precisei de tempo para a poeira baixar e eu conseguir clareza do que
gostaria de lhe dizer. Fiquei muito impressionada com o tamanho do seu
ressentimento e espero que botar para fora ajude a diminuir o sofrimento
que ele deve estar lhe causando. Lamento não ter conseguido ser a mãe de
que você precisava. Aceitar que eu tenho limitações que me impediram e
ainda me impedem de ser a mãe poderosa que eu gostaria de ter sido não é
uma coisa fácil para mim. São dez anos de análise trabalhando sobre o
mesmo tema: minha obrigação autoimposta de ser onipotente e o preço que
isso me cobrou e ainda cobra. Crescer dependendo de uma mãe que parecia
tudo poder, mas que, em muitas e frequentes situações, se sentia aguda e
sofridamente impotente não deve ter sido fácil. Imagino que eu devia ser
uma dupla mensagem ambulante. Ler sua carta serviu para confirmar essa
sensação. O~que parece emergir dela como cobrança é que eu tinha a
obrigação de defender você do seu pai e do seu irmão e não defendi
porque não quis, porque não amava você o suficiente.

Eu também acho que deveria ter defendido você melhor, mas não fui capaz.
Eu também não conseguia me defender da agressividade do seu pai. E~a
única forma que eu encontrei para sair da situação de sofrimento foi
fugir. Ainda que você não se lembre com detalhes, você cresceu vendo a
minha dificuldade de pôr limite para o seu pai e também para o seu
irmão. Só depois da separação é que eu consegui controlar a
agressividade física do seu irmão com você.

Mas acho que aí o estrago já estava feito. Sinto muito. Se um dia você
puder, me desculpe.

\begin{flushright}Mamãe\end{flushright}


\pagebreak\mbox{} 

\vspace*{.3\textheight} 

--- Mas, mãe, como você conseguiu emagrecer \emph{durante} a gravidez e
engordar na amamentação?

--- Eu te amamentava com um saco de biscoitos do lado. Eu tinha inveja
de você, eu também queria ser amamentada!

\chapter{Negações}

\vspace{-3em} 

Anna,

\smallskip{} 

Sua visão das coisas é completamente distorcida e irreal: eu nunca te
chamei de louca. Você não foi maltratada na minha casa. Te recebi muito
bem e tirei vários dias de folga para sair com você quando chegou.
Depois, você passou três semanas sozinha na minha casa e eu paguei o
aluguel! Você leva as coisas tão a sério que nem vê que quando eu digo
aos meus amigos ``Não escute o que ela diz, ela é uma comunista'' é
brincadeira. Eu não sou fascista. O~que me irrita não é a sua opinião
política, é você achar que pode ter opinião de tudo, inclusive sobre as
coisas que eu faço, sem notar o quanto me incomoda essa invasão. Você
sempre se fez de vítima. Como se o mundo te devesse algo. O~mundo não te
deve nada. Eu nunca te chamei de louca, não ponha palavras na minha
boca. Não tenho necessidade de competir com você, já expliquei, nem
quero que você fique me aplaudindo, como você diz, mas gostaria que você
respeitasse a minha competência. Não passei os últimos dezessete anos da
minha vida estudando pra nada! Você precisa deixar o passado pra trás.
Esquece a nossa infância. Sua vida começa aqui. Pra que ficar lembrando
das nossas brigas? Sua tendência a remoer o passado não leva a nada. Eu
vou tentar não levar a sério suas acusações pra não construir uma mágoa
irreparável.

\smallskip{}  \begin{flushright}Feliz aniversário adiantado,\end{flushright}

 \begin{flushright}Seu irmão.\end{flushright}


\asterisc{}

Anna,

\medskip{} 

Você continua sendo minha filha. Mesmo que eu seja o monstro que você
descreveu, continuo sendo seu pai.

Ontem, quando cheguei em casa tentei dormir preocupado com o jeito que
você saiu dirigindo daqui. Preocupado com a sensação de que eu havia
estragado mais ainda as coisas. Finalmente dormi. E~tive um sonho muito
significativo com você.

Sonhei que você estava me mostrando umas reformas que estava fazendo no
meu sítio. Havia um grande buraco quadrado de concreto que era uma
fossa, muita gente trabalhando e merda para todo lado. Essas pessoas
pareciam escravos, trabalhando muito e completamente mudos. Todos nós
éramos da cor marrom esverdeado, cor de merda, e você colorida, normal.
Como num filme do impressionismo alemão. Todo mundo, inclusive eu,
estava sujo de merda. Menos você. Você andava feliz pela obra sem sujar
nem mesmo os sapatos. Mas, curiosamente, nada cheirava mal. Eu tentava
te acompanhar mas tinha medo de escorregar na escada de concreto, também
cheia de merda. Num momento tivemos que atravessar sobre esse buraco. A~ponte era uma porção de tabuinhas frágeis, muito sujas, estendidas de um
lado ao outro. Embaixo delas um buraco imenso, profundo, cheio de merda.
Você passou feliz (em pé) rapidamente sobre as tábuas e eu tentava
passar (de quatro) morrendo de medo de escorregar ou de que elas
quebrassem com meu peso. Pavor de cair lá embaixo. Acordei muito
assustado.

Anna, vou repetir: amo meus filhos.

Com ou sem defeitos. Amo você e quero te ver muito feliz. Só não sei o
que fazer neste momento. Espero que um dia você mude a imagem de monstro
que tem de mim. Espero que um dia volte a espontaneidade que havia entre
nós, que a gente possa ter novamente a cumplicidade e o afeto que deve
haver entre pai e filha. Vou continuar esperando por isso. Nem que tenha
que esperar a minha vida toda. Nós merecemos. Seja feliz na sua vida,
seja feliz nos seus afetos.

É o que eu desejo para você.

\medskip{}  \begin{flushright}Seu pai.\end{flushright}


\chapter{À deriva}


\begin{verse}[\versewidth]
Eu só concilio o sono\\
Nos braços de algum amigo\\
Novo ou antigo\\
Novo
ou antigo\\
E o sobressalto me acorda\\
É esse viver na borda\\
Que sempre
me desconcentra\\
De peito apertado peço\\
Aos amigos que me apertem\\
E
entre os braços dos outros\\
Sufoco bem e ruim\\
E vem um sono sem
sonhos\\
Em travesseiros sem fronha\\
E uma noite sem fim\\
 Nos braços de
algum\\
 Amigo novo\\
 Ou antigo\\
 Novo ou antigo\\
 Novo ou antigo\\!
\end{verse} 

\chapter{Prerrogativas}




Anna,

\medskip{} 

A ideia de abrir a relação sempre esteve presente para mim também, achei
em diversos momentos que isso ia acontecer com alguma naturalidade em
uma troca de casais ou coisa do gênero. Não estou puto pelo fato de você
ter proposto isso. Você criou um castelo de que eu não suportaria, que
isso seria o fim da relação etc. Acho isso absurdo, tanto é que na
primeira vez que você sugeriu eu prontamente aceitei --- com as
ressalvas de que isso poderia ser o fim da nossa relação, que tudo
deveria ser feito com cuidado, respeitando os tempos etc.

A forma como sua história com o Vicente, ou com qualquer um, se acelera
no contato com a minha história com a Laura não tem nada de natural ---
existe sim uma competitividade que se instaurou no meio do processo. Por
que você não pode esperar que eu me envolva com alguém primeiro? Por que
você não espera o tempo da minha relação com a Laura?

\medskip{}  \begin{flushright}Amor, Paulo.\end{flushright}


\asterisc{}

\pagebreak{}

Anna,

\medskip{} 

Só consigo sentir vergonha, e o pior é que é uma vergonha anestesiada.
Ontem quando você chorou vi que você sofria como nunca, mas não consegui
compartilhar da sua dor como deveria. Estou assustado. Não quero que
você saia de casa, isso me deixa inseguro: não consigo assimilar o fato
de que abrir a relação é algo que possa ser mais importante do que a
nossa relação para você. Não quero que você saia, mas ao mesmo tempo não
consigo deixar de sentir uma insegurança brutal, fruto dessa mudança.
Sei que não estou sendo justo, mas não vejo como tirar forças para ser.
Quando te disse que queria destruir tudo, estava pensando em uma forma
de reverter isso: achei que pelo sexo talvez desse, por isso forcei a
barra pra transar com você, mesmo você dizendo que assim eu ia destruir
tudo o que tínhamos. Por isso insisti pra transar sem camisinha e
desdenhei quando você disse que o útero era seu, que quem teria de fazer
um aborto era você. Não sei o que fazer, sei que te amo, sei que não
quero te perder, mas não consigo reagir, não consigo sair da oscilação
inegurança-ódio.

\medskip{}  \begin{flushright}Com amor, Paulo.\end{flushright}


\asterisc{}

Anna,

\medskip{} 

Você poderia ter me dito lá embaixo que ia para a casa da sua mãe --- eu
já estava esperando. Isso significa o quê? É só um tempo? É uma ruptura?
Você vai abrir a relação? Acho ingratidão você sair assim sem conversar.

Se você está pensando em ficar com alguém nesse tempo eu gostaria que
você passasse aqui, arrumasse suas coisas e sumisse da minha vida antes.
Você não luta, apenas desiste! Fala do medo de me olhar, do clima
abafado e só. Não sei quais são as crises reais.

Espero uma carta sua, Paulo.

…

Fiquei sem conseguir dormir com a sua ausência. Foi inevitável em algum
momento pensar que me enganar seria fácil. Sofro muito nesses momentos e
desconfio de você, o que me faz sofrer mais. Fiquei perdido, me
descontrolei, forcei a barra pra transar com você, mas em momento nenhum
virei seu pai. Minha agressividade estava lá sim, era justa, estava
reivindicando algo que você me tirou. Sua buceta é minha sim! Foi minha
esse tempo todo, e você querer mudar isso agora é um desrespeito a toda
nossa história, a tudo que a gente construiu.

\medskip{}  \begin{flushright}Amor, Paulo.\end{flushright}


\asterisc{}

Anna,

\medskip{} 

Fiquei muito decepcionado em te encontrar. Confirmou a minha expectativa
de que você está povoando um outro universo, com outros hábitos, outros
valores. Um universo que para sua sorte, ou para seu azar, joga todo a
seu favor. Um universo onde sua voz destoa, onde seus gestos são falsos,
onde tudo vira afetação. Um universo onde o analista pode ser amigo e
apoiar seus desvarios, onde as pessoas que você escolhe para se
relacionar, por serem jovens, te veneram. Fui para a defesa da tese do
Pedro com alguma esperança de não confirmar minhas intuições, esperando
encontrar a Anna que conheci.

Quando digo que venho de outro lugar estou querendo dizer que nasci em
um lugar igual ao seu e que perdi na adolescência esse lugar --- você
conhece a ruína, em todos os sentidos, que é casa dos meus pais. Nunca
ganhei um carro, por exemplo, você ganhou dois. Nunca tive mesada, curso
de inglês, cursos de dança etc. pagos pela família. Tive algumas
regalias (como toda a ex-classe média) até meados dos anos 80, depois
foi ruína. A~sua mãe sempre teve as contas em dia, morou em um belo
apartamento e teve tranquilidade para se formar e fazer o que gosta.
Você vai ser herdeira, sua mãe tem condições de te ajudar.

E agora essa palhaçada máxima do seu direito ao orgasmo: anos de
construção juntos, de empenho amoroso, para você pegar e jogar tudo fora
em nome de um orgasmo que nunca tinha tido? Sinto dizer, a coisa pode
estar gostosinha, confortável, mas está girando em falso em um mundo
pequeno-burguês. Quem te dava dado de realidade, quem fazia você atritar
com o mundo, quem colocava o tempo todo o que você pensava em xeque era
eu --- pergunte para os nossos amigos. Quem surtou fora de qualquer
medida foi você, e infelizmente você ainda está longe de perceber isso.
Procure os amigos (os nossos, os de verdade), converse com eles --- e
troque de analista.

\medskip{}  \begin{flushright}Amor, Paulo.\end{flushright}


\asterisc{}

\pagebreak{}

Anna,

O fato de eu nunca ter te elogiado nesse nove anos não tinha um sentido
de te diminuir, de te deixar se sentindo feia, pois esse é seu ponto
fraco etc. Era só uma limitação minha que eu sempre tentava compensar
com afeto, com carinho, com dedicação, com admiração etc. Acho que você
é excessivamente vaidosa, precisa de elogios o tempo todo.

Sua ``mutilação'' --- acho um exagero --- não é fruto daquela noite e
dos desencontros que tivemos ali (como você sentiu estar se oferecendo
para um estupro, quando eu só tentava reconstruir nossa relação pela via
sexual?!). É~natural que você agora não consiga transar: eu e você vamos
demorar um tempo para retomar nossa sexualidade, para tomarmos posse
dela. Aceite um pouco essa condição ``mutilada''. A~mutilação vem da
ausência do outro, e não daqueles momentos de agressão que vivenciamos
da maneira que pudemos.

Queria pedir desculpas pelas acusações feitas em momentos de surtos
ciumentos, não estava conseguindo segurar minha onda. Nunca quis te
destruir, mesmo quando eu disse que queria.

\asterisc{}

Anna,

\medskip{} 

Nunca te proibi de entrar aqui, posso ter falado isso em um momento de
raiva, mas você sabe muito bem que eu nunca sustentaria isso, não seja
paranoica. Com as cartas tentei manter contato, era uma tentativa,
desajeitada muitas vezes, irritada outras, de entender as coisas para
tentar recompor nossa história. Você chegou à conclusão que aquilo não
levaria a nada (que era ``espancamento moral'' etc.) e rompeu o contato
comigo por mais de cinco meses por causa do mestrado! Agora você quer o
que, que eu diga vem cá meu amor, pode pegar seu diploma para fazer sua
defesa, deixo a chave na portaria? Isso não, pode esquecer! Você vai ter
que me ouvir sim!

Acho que a nossa história merecia, pelo menos, que você tivesse perdido
os prazos, perdido o mestrado. E~você sabe muito bem que eu \emph{nunca}
teria ficado com a Laura se você não tivesse me empurrado tanto para
isso.

\medskip{}  \begin{flushright}Paulo\end{flushright}

\bigskip{} 

P{}.S{}. Estou terminando de arrumar os livros e o resto das roupas,
mas os discos não faz sentido você levar --- você não tem onde ouvir
ainda.

Acho que mesmo que eu tenha tido inveja e tentado te deixar mal nas
vezes que você me ligou, sua reação não se justifica. Eu não sou outra
pessoa, estou apenas com raiva pela forma como as coisas aconteceram.

\asterisc{}

Anna,

\medskip{} 

Como já disse outra vez, toda construção que você fez está apoiada em um
processo alucinatório que é fácil de ser quebrado para qualquer um que
esteja fora. O~fato de eu ter tido crises de ciúmes e ser violento não
me converte no seu pai --- nem que eu quisesse me transformaria nele.
Você acha, na sua construção, que eu já estava ``louco'' antes e um dos
argumentos que você dá é o fato de eu ter insistido para a gente transar
sem camisinha. A~sua hipótese de eu estar querendo ali uma prova de amor
é bem plausível e, como não tenho outra melhor, aceito essa --- às vezes
estamos frágeis e precisamos delas. Mas daí a virar o seu pai tem chão.
Sei que na sua cabeça eu peguei no seu ponto fraco para te destruir, mas
mesmo aceitando isso, que tem muito de disparate, o que aconteceu não se
justifica e você em alguma medida sabe disso quando precisa projetar seu
pai em mim. Você disse ao telefone algo mais ou menos assim: é triste
saber que o nosso amor estava ancorado na exclusividade sexual. Mas
estava, era um apoio forte da relação e sabíamos, ou intuíamos isso
quando conversamos sobre abrir o casamento --- sabíamos que isso podia
levar ao fim, como levou. O~que é verdadeiramente triste para mim é
reduzir o amor ao desejo sexual. Não recuar do seu propósito de
``liberdade'' foi fazer essa redução. Ainda mais sem permitir as minhas
crises de ciúmes, taxando todas de machismo, de violência contra a
mulher etc. Acho, e continuo achando, que se você recuasse, se tivesse
conseguido redimensionar nossa história, as coisas teriam sido
diferentes. Mas não foram. Você disse que não tem volta, essas são
minhas demandas, elas são a verdade do mundo, uma escolha política. Se
for assim estamos fritos: o amor está reduzido a contatos eróticos. As
coisas deveriam valer dentro da nossa história, contraditória,
problemática --- e não nessa abstração boba do tipo ``relação aberta é
algo mais verdadeiro ou menos verdadeiro que a monogâmica''. Você parece
uma revista feminina para a classe média!

Você disse várias vezes que tinha a esperança que eu admitisse ou
enxergasse a violência que fiz a você: na verdade você está querendo que
um dia eu admita que eu estava agindo como seu pai naquela situação ---
só que isso foi um processo projetivo criado por você a partir de
situações verdadeiras de ciúmes que muitas vezes são violentas mesmo.

Acho que você se atrapalhou e botou tudo a perder, ou não me amava mais
--- sei lá, tanto faz agora. Você fez a sua escolha: sair de casa. Não
te expulsei, como você insistiu várias vezes, nem vou te deixar sem
qualquer coisa que você julgue sua, como você também está insistindo em
acreditar apesar de todas as provas na direção oposta. A~situação ficou
insuportável para você muito por conta da sua projeção que a remeteu à
sua história --- verdadeira ou construída, tanto faz, de opressão.

\medskip{} 
\begin{flushright}Paulo\end{flushright}


\asterisc{}

Anna,

\medskip{} 

Sei que você ainda vai argumentar que minha verdade está muito lisa e
que a verdade toda --- muito mais lisa, diga-se de passagem --- é que eu
te troquei pela Laura. Aceitei muitas das suas verdades sem reduzir umas
às outras, só não abrindo mão do fato irredutível de você ter projetado
seu pai em mim. Eu ter te trocado por outra, como você bem notou, é
verdadeiro, se coloca como um fato presente, mas essa decisão aconteceu
depois de você dizer que o nosso ``exclusivismo'', que alicerçava sim a
relação, não tinha mais volta --- ou, para me repetir, depois de você
ter quebrado o nosso pacto amoroso, depois de você mudar as regras no
meio do caminho em um momento de extrema fragilidade minha. Acho sim que
isso tem muito de egoísmo. Eu queria, como disse várias vezes, ter
outras histórias de amor, não apenas satisfazer o desejo do solitário
--- isso faço muito bem com masturbação. E é essa tentativa de amor que
estava tendo com a Laura e que acredito teria ficado naquele patamar sem
sexo, o que tanto faz, de qualquer forma. Em alguns momentos vislumbrei
essa possibilidade radical de ter dois amores, até me dar conta em
análise que um amor já requer muito da gente para ser verdadeiro e como
você estava irredutível em relação às suas demandas mergulhei nessa
história. Nada aqui foi maquinado ou maquiavélico, ou tem fundos
sinistros como você quer acreditar. Também não usei ela de ponte para
sair da nossa história, se fosse isso talvez não estivéssemos mais
juntos --- mas de qualquer forma tanto faz, isso agora é problema meu .
Só me dei conta em determinado momento --- em análise --- que não tinha
como suportar suas demandas no presente, nem como uma bomba relógio que
estouraria para frente de maneira inevitável como você cansou de dizer.
Isso seria, para usar uma boa expressão sua, vida ruim para mim e para
você. Acho que o desejo é sim verdadeiro, mas para mim não chega a
arranhar o amor quando eleé verdadeiro --- como disse, para me repetir
novamente, para mim masturbação resolve. Se você aceitasse esse fato sem
achar que eu sou um filho da puta as coisas ficariam mais fáceis. Mas o
central mesmo para continuarmos, caso você queira, tendo uma relação com
algum patamar de verdade, é você trabalhar para entender a projeção do
seu pai em mim para poder se relacionar com um Paulo, que não é mais seu
marido, mas que de jeito nenhum é seu pai, verdadeiro ou fictício.

Por que será que é tão difícil para você entender que eu fiquei com
ciúmes, inseguro etc., e isso tem consequências na direção da ``posse''
do corpo do outro, no desejo, na agressividade etc. Você o tempo todo
afirmou que era de matéria diferente do mundo, que lidava bem com o fato
de eu transar com outra mulher, ou amar outra mulher etc.

Não gostei nem um pouco da acusação de irracionalidade: eu tento ---
muitas vezes de forma desajeitada --- apontar para processos concretos
dizendo ``aqui você distorceu'', ``aqui você exagerou'', ``aqui você
tinha razão'' etc. E~você, leia suas cartas e veja se elas são dignas de
acabar uma relação como a nossa, ou vaga, ou muito abstrata, ou em
silêncio. As minhas são patéticas, mas estão na luta, tentando entender
as coisas. Agora o troço está mais claro para mim --- não resolvido ---
e você não baixou ainda para as coisas concretas --- e o pior é que
talvez nem baixe --- protegida por, primeiro, a minha violência que não
te deixava respirar (não foi sua projeção do seu pai nas minhas crises
de ciúmes…), depois o fato, paranoico, que eu troquei você pela
Laura, já que --- olha sua prova, ou sua ``racionalidade'' --- eu estou
com ela. Francamente, qualquer pessoa que você diga isso comprará sua
versão --- é a explicação clichê de por que as relações acabam --- mas
se você minimamente contar o que rolou mesmo, organizando no tempo,
racionalmente, fica difícil de apontar isso como causa do fim.
Sinceramente não entendo o que você entende por racionalidade e
irracionalidade, para mim isso significa descer às coisas com risco de
voltar sem as certezas de antes, como fiz várias vezes ao longo das
cartas.

Olha, sua pressa de terminar rápido a conversa está querendo dizer algo
--- também tenho pressa, mas ela não pode atropelar os tempos do
processo. Sinceramente, sete meses não são porra nenhuma para uma
relação de nove anos como a nossa foi. Tenho um pouco de receio dessa
sua pressa, acho-a pouco reflexiva. Não se vira a página e segue em
frente, sem elaborar as coisas.

\begin{flushright}Paulo\end{flushright}


\chapter{Capital flexível}

Anna,

\medskip{} 

Você ficar ``bem'', ou se recompor e afirmar um narcisismo teoricamente
nunca vivido, custa o meu ficar mal --- a verdade dessa dinâmica mostrou
sua cara toda bem recentemente: o seu ficar bem pressupõe o meu ficar
mal, o ficar bem narcísico não suporta o outro. O~outro só existe na
superação do narcisismo. O~narcisismo é o grande câncer atual, como você
bem sabe, é a doença do nosso mundo sem outro pós-industrial. A~sua
resposta a isso é: e o seu narcisismo? E a tetinha da mamãe e tal? Acho
que a resposta para isso é: a minha sensação de perda, com a qual não
estou sabendo como lidar, é de perda do outro e de um outro constituído
na base de amor, com superação, com dedicação etc., mas dentro de uma
chave monogâmica. A~base desse amor se transformou com esse dado recente
do narcisismo: narcisismo e amor são incompatíveis: a diferença é o
``outro'', no caso o bocó aqui. Nosso amor não existe mais, pelo menos
do seu ponto de vista, se o quadro é esse mesmo. Enquanto você não
superar esse narcisismo que é uma forma estranha de ``superação'' da sua
história acho que não temos o que fazer juntos. Vamos apenas conseguir
ficar mais machucados e um só irá potencializar o sofrimento do outro,
pois eu sou um ``outro'' no seu narcisismo e você é um não-outro no
nosso amor --- que obviamente não tem como existir nessas bases. Essa
mudança de base no nosso amor, ou, para usar os termos com a
radicalidade que você estava pedindo, esse amor que não tem mais base
alguma, é o que me leva a todos os fantasmas que estão vindo à tona
agora. Vou enfrentá-los com todas as forças e meios que tiver --- vou
fazer análise, vou pensar e escrever sobre isso o tempo todo etc. Mas,
como você já percebeu, vai ser uma luta quixotesca --- ou mais uma ---
pois a situação toda está montada pela sua condição, atitudes etc.

Minha agressividade, que estava lá, sim, era justa na medida em que
estava reivindicando algo que foi tirado de lá, tinha uma direção que o
tempo todo foi afirmada: não deixar as ``nossas'' conquistas recuarem.
Achei que se não fizéssemos um esforço afetivo e sexual a coisa poderia
piorar. Depois mais uma vez fui quixotesco: a minha vontade de
destruição naquele momento estava direcionada contra moinhos de vento
--- como destruir algo que não estava mais ali? Acho que era isso que
queria testar ali, queria que a realidade espatifasse a minha cara e
mostrasse que o que estava sentindo e vendo não era real. Agora ficou
claro para mim: toda a incerteza em que me encontro, e todo ciúme,
insegurança etc. são no fundo certeza de que o nosso amor não está mais
lá, que meu chão não está mais lá. Anos de construção coletiva, de
empenho amoroso, de apagar os contornos da sua masturbação para
incorporar em uma relação com um outro, para você pegar e apagar nossa
história e transformar suas demandas em um absoluto, em algo que é maior
e mais determinante que o outro sujeito dessa construção --- que o outro
da sua história. Não sei o que você fez com as suas leituras --- a coisa
principal que fica em qualquer leitura de teoria crítica, ou coisa que o
valha, é o fato das coisas terem história, e sempre precisarem ser
consideradas assim para que não girem em falso.

Você sabe que gosto do seu analista, acho que ele é uma pessoa doce que
foi fundamental no momento mais difícil da sua vida --- deu toda a base
de amor necessária para o processo terapêutico começar, o que é
louvável. Mas, por outro lado, a recomposição amorosa da sua história
--- que é um dos objetivos da terapia --- te levou a fundar um
narcisismo. Aqui dou um passo atrás: é papel do terapeuta também castrar
--- e papel central e decisivo para a história do paciente. Não conheço
o miúdo do processo, nem quero, não sei se foi a qualidade da sua
depressão --- seus contornos melancólicos --- que a levaram para aí. Não
sei, na minha ignorância psicanalítica, se em um casos desses a pessoa
precisa se recompor em um narcisismo antes de voltar ao outro ---
realmente não sei. O~que sei é que meu lugar no mundo acabou e você está
em uma roubada: o narcisismo é uma doença que encontra com tranquilidade
objetos no mundo para mascarar a falta do outro. Aliás, o mundo no atual
estágio do capitalismo é um playcenter para narcisos. O~que quero dizer
é que minha vida é essa, encaro as coisas do jeito que posso, mas
encaro. Não fui tão deformado pelo mundo a ponto de acreditar que o que
desejo, o que acredito como verdadeiro, seja imediatamente aplicável só
porque penso assim. Você não está realmente se curando, desde quando ter
orgasmo é se curar?! De novo o engodo da hipertrofia psicanalítica. Você
irá se curar quando adquirir uma bossa reflexiva para que os problemas,
quaisquer que sejam, parem de te aniquilar. Desculpe-me se fui impreciso
do ponto de vista da psicanálise. Do ponto de vista hegeliano, a tomada
de consciência é sempre um deslocamento de um lugar que se entendia como
verdadeiro --- o que necessariamente acarreta uma dor quando a
consciência topa enfrentar: uma forma de castração da consciência
imediata em função da mediada. É~isso que estava pensando como o papel
do psicanalista, estou errado?

Esse mundo de terapia, mestrado, fora do mercado de trabalho, faz com
que tudo que carregue uma verdade maior seja colocado de canto --- por
ser agressivo, rude, áspero (não verdade abstrata, verdade desse tipo
que está em uma rede de construção intersubjetiva, complexa, com outros
fazendo parte dela etc.) --- aspectos geralmente constitutivos dessas
verdades mesmas. Tirando esse incômodo --- eu --- você está leve e solta
para o filistinissmo --- boa sorte! Com certeza é a escolha mais
confortável. Só me deixe fora disso. Não vou participar desse semimundo
acadêmico, nunca foi minha praia. Se você ainda for capaz de aguentar
essa agressão --- não vivo no mundo dos teletubies, no meu mundo quando
se acredita em algo se age com agressividade para transformar pessoas e
coisas --- fique sabendo que estou disposto a ser o mais honesto
possível com você --- como disse, não virei outra pessoa, mesmo você
querendo que isso acontecesse. Eu não participo e não vou participar do
mundinho acadêmico, se você quiser vá em frente. Só estou lá a passeio e
para estudo --- esse não é meu mundo, eu só vivo nele.

Estou bem, a terapia tem me feito muito bem --- tomo várias invertidas
por consulta, como acho que tem de ser, por isso comprei a coisa mesmo
desempregado \mbox{---,} você não precisa vender seu carro para isso, mas de
qualquer forma muito obrigado pela oferta. Mesmo que você me banque,
coisa que você sabe que não vou aceitar (orgulho de classe), não vou ter
alguém que leia o que escrevi quando chegar em casa, que ajude a
destravar os pepinos, que pense a coisa junto, que dê um conforto
material e afetivo cotidiano --- será simples valor de troca que não
aceitarei. Estava pensando em coletivizar tudo que tenho com você, esse
era meu plano até te encontrar --- ou não te encontrar.

Fiquei esse tempo todo com você sendo idealizado, você nunca me olhou
como um ser limitado, com problemas --- pelo menos nesse aspecto
afetivo-sexual --- no momento em que essas limitações mostram o rosto
você pega e me acusa falando que sou um macho sórdido, Bentinho (nunca
mais me chame disso, pois não conseguirei mais olhar na sua cara ---
isso vai significar o seu desaparecimento da minha vida) etc. É~estranho
uma realidade que se move de forma tão contrastada, você não acha? De
príncipe encantado a Bentinho? A verdade está em que lugar? Talvez ela
esteja nessa oscilação radical que molda a sua forma de olhar para a
realidade, desse deixar vivo de forma imediata --- não reflexiva (aquilo
que nos faz sujeitos) --- todo seu passado em uma chave melancólica. Seu
passado invade seu presente o tempo todo --- como nos romances que a
gente estuda --- e não se coloca como experiência, se coloca como
realidade imediata. Você nunca teve como achar uma instância reflexiva,
se é que é possível, nessa melancolia que te constitui. O~tempo fora da
realidade brutal que é o mercado de trabalho acentuou os contrastes e,
consequentemente, seu narcisismo --- você foi quase para um universo
paralelo, onde tudo funciona como ideal purificado de qualquer mescla,
de qualquer dúvida, de qualquer limitação. Não fui mesquinho nem
controlador, só apontei essa contradição (que funciona nesse mesmo
mecanismo melancólico que estou amorosamente tentando explicitar, com
bastante esforço e dor, para trazer a Anna de volta ao mundo físico).

Foi muito difícil para mim, e me deu muita raiva, o fato de você ter
ficado com medo de mim, não conseguir olhar para mim, sentir asco do meu
cheiro etc. Isso acabou comigo naquele contexto. Desculpe-me por me
estender tanto, eu sei que você está ocupada terminando o mestrado, mas
acho que pode ser uma forma da gente não quebrar todos os laços.

Você insiste em não ver o mundo ideal em que está vivendo. Outra coisa,
para que a agressão? Acho que nos tornamos mesmo outras pessoas, mas não
quero sair daqui: me sinto mais com a minha cara, com o meu cheiro, com
a minha verdade --- realizando meu conceito que continua nos mesmos
pressupostos que você conhece --- e você? E o seu desejo narcísico sem,
como é claro, outro? E você me conduzir rumo a minha destruição e ficar
só repetindo ``não tem volta''? Não tem nada perverso aqui? A realização
do meu conceito continua nessa direção, tentando ser mais doce, mais
honesto em relação aos meus sentimentos, sabendo demonstrá-los com mais
naturalidade etc. Isso aconteceu em boa medida com você até a minha
desestruturação promovida pelo seu narcisismo. Enfim, continuo na
disposição de ser seu amigo e solidário até o fim, mas não quero mais
outras formas de amar, quero algo bem mais modesto: me reestruturar e
tocar minha vida e minhas escolhas.

\medskip{} 
\begin{flushright}Paulo\end{flushright}


\asterisc{}

Anna,

\medskip{} 

Achei o \emph{Eros e civilização} aqui em uma pilha que fiz e esqueci de
colocar nas malas que mandei. Separei os outros livros que achei (o Caio
Prado e o Sérgio Buarque). Você pode devolver os meus já?

Fiquei chateado pelo fato de nos agradecimentos eu ter ficado reduzido a
um apoio, quase como se agradece a uma bolsa. Eu teria preferido, e
acharia mais honesto, você ter agradecido o apoio na estruturação da
coisa --- ou no acompanhamento quase ombro a ombro, como você sabe que
foi. Não há nenhum tipo de indicativo de que eu te ajudei a pensar o
troço!

Gostaria que você, agora que está com emprego e o mestrado resolvido,
ficasse com os gatos. Vai me doer o coração me separar deles, mas me dói
mais deixá-los aqui por causa do espaço e pelo fato de lembrar de um
projeto de vida e amor interrompido --- acho que com esse ``custo'' você
tem que arcar. Além deles me lembrarem o tempo todo que não fui capaz de
adaptar meu corpo às novas demandas (nem tão novas assim, pois elas têm
a nossa idade mais ou menos) do capital flexível. Eu, como Trotsky e os
marxistas, infelizmente vivo na tradição. Acho que sou mesmo cuecão e a
terapia está me fazendo ver isso com tranquilidade --- sem abstratamente
achar que uma forma é superior a outra. Você tem demandas de um tipo ---
ser desejada o tempo todo etc. --- e eu quero um amor monogâmico. Ficou
claro para mim que a monogamia é mesmo uma forma muito limitada de amor
para quem precisa ser desejada de quinze em quinze minutos.

Toda relação se estabelece com algum grau de dependência: Freud
dinamitou a pretensa ``autonomia'' do sujeito forjada no século \versal{XIX}. A~nossa dependência amorosa aconteceu com um grau altíssimo de autonomia
dos dois --- quase um ideal (ou eu estou realmente louco). Mas você
exagerou tudo --- como é característico do psicótico depressivo, segundo
o Winnicott --- e fez o que fez. Mas no fundo o jogo já tinha acabado,
acho que não conseguiria mesmo dar a volta na história e também não
conseguiria viver com o tipo de represamento --- ou acolchoamento ---
das paixões que você exigia: foi minha primeira crise de ciúme em nove
anos! Acho, especulando, que isso tem a ver com o fato de você ter
sofrido assédio sexual na infância. O~Winnicott diz que é quase
impossível começar um processo terapêutico com um paciente assim, pois
ele impede o acesso onde melhor funcionariam as coisas. Ponto para o seu
analista que foi brilhante aqui mas depois, acho, se acomodou com essa
vitória que não foi pouca --- não cumpriu o papel de situar, justamente
o de pai (verdadeiro). Digo isso pelo fato dele não ter chamado sua
atenção para a projeção brutal e caricata do seu pai em mim --- um dos
fatores do nosso fim, talvez o principal. É~bom lembrar que analista é
humano e faz merda). É~bom lembrar também que a psicanálise começou
quando Freud passou a duvidar dos depoimentos das suas histéricas, e que
qualquer processo analítico começa nesse contrapé, no momento em que a
realidade se põe nesse patamar contraditório.

Você teve razão em muita coisa que disse no último telefonema. Em
relação à minha atração pelos estados psicóticos, isso é verdade. O~Winnicott, de novo, dizia que viver só na neurose é muito chato, é
preciso flertar com esses estados psicóticos para ter saúde. Coisa que
você faz o tempo todo: tem sempre no mínimo dois amigos deprimidos. Não
estou falando isso para te recriminar, não quero que você se ofenda com
nada que estou escrevendo aqui. Quero que você tente pensar nisso, pois
disso depende sustentar alguns destroços da ruína.

Não precisava escrever nada disso, fiz com carinho e por acreditar que
devia me posicionar para tentar te situar (sem opressão) dentro do que
eu acredito como verdadeiro. Não me interessa ``ganhar nada'', já
perdemos os dois, como já disse. Agora o que sobra é tentar criar alguma
consciência da ruína e aprender, em mais um plano, a viver de cabeça
erguida na derrota tentando preservar o que é essencial.

\medskip{} 
\begin{flushright}Com carinho, Paulo.\end{flushright}



